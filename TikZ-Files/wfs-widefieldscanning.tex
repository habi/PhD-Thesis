% !TEX root = ../Thesis.tex
%\documentclass{article}
%\usepackage[pdftex,active,tightpage]{preview}
%\usepackage{tikz}
%\begin{document}
% \def\scale{1}
%\begin{preview}
	\begin{tikzpicture}[thick,scale=\scale]
		% rotation axis
			\draw[->] (0,-3) ++ (-50:.75) arc (-50:300:.75 and .25);
			\draw[] (0,-4) node[below] {Rotation axis} -- ++(0,3);
			\draw[dotted] (0,-1) -- ++(0,2);
		% position 1
			\draw (0,-1) circle (1.5 and .5);
			\fill[shade,nearly transparent] (-1.5,-1) arc (-180:0:1.5 and .5) -- ++(0,2) arc (0:180:1.5 and .5) -- cycle;
			\draw (-1.5,-1) arc (-180:0:1.5 and .5) -- ++(0,2) arc (0:180:1.5 and .5) -- cycle;		
			\draw (-1.5,1) arc (-180:0:1.5 and .5);
			\draw (1.25,-1.5) node[right] {1\textsuperscript{st} Position};
		% position 2
			\draw (0,-1) circle (4.5 and 1.5);
			\fill[shade,nearly transparent] (-4.5,-1) arc (-180:0:4.5 and 1.5) -- ++(0,2) arc (0:180:4.5 and 1.5) -- cycle;
			\draw (-4.5,-1) arc (-180:0:4.5 and 1.5) -- ++(0,2) arc (0:180:4.5 and 1.5) -- cycle;		
			\draw (-4.5,1) arc (-180:0:4.5 and 1.5);
			\draw (2.25,-2.7) node [right] {2\textsuperscript{nd} Position};
		% top from position 1 on top
			\draw (0,1) circle (1.5 and .5);
		% rotation axis on top
			\draw[->] (0,1) -- ++(0,2.5);									
		% sample movement
			\draw [red,ultra thick,->] (0,1) -- (3,1) node [fill=white,semitransparent,text width=3cm,anchor=west] {Sample movement relative to beam and camera} node [black,text width=3cm,anchor=west] {Sample movement relative to beam and camera};	
	\end{tikzpicture}
%\end{preview}
%\end{document}