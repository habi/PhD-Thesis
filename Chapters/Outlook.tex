% !TEX root = ../Thesis.tex
\acresetall
\myChapter{Outlook}\label{ch:outlook}
\begin{flushright}{\slshape    
		So Long, and Thanks for All the Fish} \\ \medskip
    --- \defcitealias{Adams1984}{Douglas Adams}\citetalias{Adams1984} \citep{Adams1984}
\end{flushright}

\vfill

The tomographic images used in this work served as a common ground truth to obtain insights into the mammalian lung. Morphometric parameters of the terminal airways obtained on three-dimensional reconstructions of tomographic data have been used to examine structural and functional parameters of the gas-exchange region in rat lung samples.

The wide field scanning method presented in \autoref{ch:haberthuer2010} is used for routine measurements in our group. The current workflow uses custom \acsu{matlab}-scripts to merge the projections from the several recorded subscans. The goal for the next few months is the integration of the wide field scanning method in the image processing pipeline~\cite{Hintermueller2010} of \acsu{tomcat}, especially into the software used to convert the projections to sinograms, called \texttt{Prj2Sin}. \texttt{Prj2Sin} already integrates the standard \SI{360}{\degree} off-center sample rotation method to double the horizontal field of view (as shown in \autoref{subfig:wfs360}). For such a \SI{360}{\degree} scan all projections are acquired in one consecutive sequence, and through simple re-ordering of the projections on the disk \texttt{Prj2Sin} can compute the overlap between the images and calculate the merged sinograms. For a \ac{wf-srxtm} scan we generally acquire three separate subscans in three different directories. At the moment, the merging of one dataset with totally \num{10734} projections from three subscans takes a bit more than one hour to perform using the \ac{matlab}-scripts. Stitching the projections directly with \texttt{Prj2Sin} instead of performing the overlap-search, reading and merging the original projections and writing the merged projections to disk prior to the generation of merged sinograms would greatly speed up the processing from acquired projections to merged sinograms to reconstructed tomographic wide field scan slices. Integrating the workflow into the image processing pipeline would both decrease the merging time and make the whole process available to other users of the beamline as an easy to use software package.

The feasibility of the enlargement of the horizontal field of view of tomographic endstations has been demonstrated and the resulting datasets have been used for the analysis of large volumes of the terminal airways with ultra high resolution. Future work will focus on the use of \ac{wf-srxtm} dataset for the extraction and analysis of the acini in the lung, as presented in \autoref{subsec:stereological analysis}. The presented method which uses so-called manhole covers to separate single acini from a central airway has been applied to multiple rat lung datasets recorded over the course of the postnatal development from day four to day 60. The definition of the location of the manhole covers in the airway tree is a process relying on manually locating the position of the junction in the airway based on morphological criteria and subsequent drawing of the manhole cover at these locations.

Using the airway structure obtained from a prior skeletonization process as a guideline for the placements of these manhole covers is planned for the future. Such a guideline would both speed up the process of locating the exact positioning of the manhole covers and reduce the needed a priori morphological knowledge of the airway tree. Combining the aforementioned manhole cover method with a skeletonization of the airway tree would make it possible to extract and segment large numbers of acini for the analysis in relatively short time.

Currently achieved results hint towards a larger change of the acinar size during lung development than previously assumed. The results achieved up to now hint towards a clear distinction between the conducting and the gas-exchanging airways (as described in \autoref{sec:functional units of the lung}) for days 4, 10 and 21, while for days 36 and 60 the change between those two points seems more gradual. Since the segmentation of single acini at the time points of day 36 and 60 after birth is much more complicated than at the earlier time points, additional work is needed to clarify the results. Segmenting more acini on the already present datasets will help to describe this change in more detail. 

Additionally, we plan to use the obtained \ac{wf-srxtm} datasets for the extraction of larger segments of the terminal airways than currently available. These extracted regions will be used to improve existent models of airflow inside the terminal airways \cite{Sznitman2007,Sznitman2009} using \ac{cfd}. 

Using the STEPanizer (described in \autoref{sec:stepanizer}), we plan to stereologically analyze the tomographic datasets in more detail. With the first results presented in \autoref{ch:discussion} we have shown that---as expected---analyzing tomographic data yields results which match stereological results obtained from classic histological sections. Performing the stereological analysis on virtual tomographic slices is advantageous compared to performing the analysis on real histological slices, since tomographic data can be recorded without destroying the sample and gives a fully unrestricted three-dimensional view inside the sample in addition to the stereological assessment.

The work on the multimodal imaging approach presented in \autoref{ch:xrm2008} is carried on by our former Master- and now Ph.D.-Student Sébastien Barré; he is developing a method to automatically register two different multimodal datasets as presented in \autoref{sec:multimodal imaging}.
\bibliographystyle{unsrtnat}
\bibliography{../../references}