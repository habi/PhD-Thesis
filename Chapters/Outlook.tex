% !TEX root = ../Thesis.tex
\acresetall
\myChapter{Outlook}\label{ch:outlook}
\begin{flushright}{\slshape    
		So Long, and Thanks for All the Fish} \\ \medskip
    --- \defcitealias{Adams1984}{Douglas Adams}\citetalias{Adams1984} \citep{Adams1984}
\end{flushright}

\vfill

The application of tomographic imaging permits an unprecedented insight into the mammalian lung. Tomographic imaging enables us to see and observe the object of interest to gain insight into the details of the sample to be studied. Besides the pretty images of the beautiful biological samples which spur the interest in this line of work, biological question can be scientifically answered using tomographic images as a base.

The tomographic images used in this work have served as a common ground truth to obtain insights into the mammalian lung. Morphometric parameters of the terminal airways obtained on three-dimensional reconstructions of tomographic data have been used to examine structural parameters of the gas-exchange region in rat lung samples. Further work will focus on the extraction of larger region of the terminal airway suitable to simulate and model airflow inside the terminal airways using \ac{cfd}.
 
\section{Sebastien multimodal}
\begin{itemize}
	\item Automatic registration of multimodal datasets
	\item No longer relying on ``luck''
	\item Maybe even multimodal registration of \ac{em}-tomography
\end{itemize}

\section{Structural/Morphological Analysis of the lung}
\begin{itemize}
	\item Further work needs to be done with STEPanizer for the stereological analysis of the huge amount of scanned samples we now have\ldots
	\item Acinar size analysis of \ac{wf-srxtm} data
	\item Further proof of viability of tomographic imaging, not only \cite{Tsuda2008}, which has been taken up with mixed feelings\ldots
\end{itemize}

\section{Skeletonization}
\begin{itemize}
	\item ATS2009-Poster
	\item 2\textsuperscript{nd} year examination
	\item Collaboration with MeVis or even artorg.unibe.ch?
\end{itemize}

