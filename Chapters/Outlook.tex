% !TEX root = ../Thesis.tex
\acresetall
\myChapter{Outlook}\label{ch:outlook}
\begin{flushright}{\slshape    
		So Long, and Thanks for All the Fish} \\ \medskip
    --- \defcitealias{Adams1984}{Douglas Adams}\citetalias{Adams1984} \citep{Adams1984}
\end{flushright}
\vspace{6cm}

The tomographic images used in this work served as a common ground truth to obtain insights into the mammalian lung. Morphometric parameters of the terminal airways obtained on three-dimensional reconstructions of tomographic data have been used to examine structural and functional parameters of the gas-exchange region in rat lung samples.

Wide field scanning has become a routine scanning method in our group. To make this method availabe for other groups, several improvements are planned for the future. The current wide field scanning workflow uses custom \acsu{matlab}-scripts to merge the projections from the several recorded subscans. The goal for the next few months will be to integrate the wide field scanning method in the image processing pipeline of \acsu{tomcat}, especially into the software used to convert the projections to sinograms, called \texttt{Prj2Sin}. \texttt{Prj2Sin} already integrates the standard \SI{360}{\degree} off-center sample rotation method to double the horizontal field of view (as shown in \autoref{subfig:wfs360}). For such a \SI{360}{\degree} scan all projections are acquired in one consecutive sequence, and through simple resorting of the projections on the disk \texttt{Prj2Sin} can compute the overlap between the images and calculate the merged sinograms. For a \ac{wf-srxtm} scan we generally acquire three separate subscans in three different directories. At the moment, the merging of one dataset with totally \num{10734} projections from three subscans takes more than one hour to perform using the \ac{matlab}-scripts. Stitching the projections directly with \texttt{Prj2Sin} instead of performing the overlap-search, reading and merging the original projections and writing the merged projections to disk prior to the generation of merged sinograms should reduce the time used for processing the acquired projections to merged sinograms to reconstructed tomographic wide field scan slices. Integrating the workflow into the image processing pipeline would both decrease the merging time and make the whole process available to other users of the beamline as an easy to use software package.

The feasibility of the enlargement of the horizontal field of view of tomographic endstations has been demonstrated and the resulting datasets have been used for the analysis of large volumes of the terminal airways with ultra high resolution. Future work will focus on the use of \ac{wf-srxtm} dataset for the extraction and analysis of the acini in the lung, improving the preliminary results presented in \autoref{subsec:stereological analysis}. The presented method which uses so-called manhole covers to separate single acini from a central airway has been applied to multiple rat lung datasets recorded over the course of the postnatal development from day four to day 60. The definition of the location of the manhole covers in the airway tree is a process relying on manually locating the position of the junction in the airway based on morphological criteria and subsequent drawing of the manhole cover at these locations.

Using the airway structure obtained from a prior skeletonization process as a guideline for the placements of these manhole covers is planned for the future. Combining the aforementioned manhole cover method with a skeletonization of the airway tree to act as a guideline for the positioning of the manhole covers would make it possible to extract and segment large numbers of acini for the analysis in relatively short time.

Additionally, we plan to use the obtained \ac{wf-srxtm} datasets for the extraction of larger segments of the terminal airways than currently available in order to improve existing models of airflow inside the terminal airways \cite{Sznitman2007,Sznitman2009} using \ac{cfd}. 

We plan to stereologically analyze the tomographic datasets in more detail using the STEPanizer. With the first results presented in \autoref{ch:discussion} we have shown that---as expected---analyzing tomographic data yields results which match stereological results obtained from classic histological sections. Performing the stereological analysis on virtual tomographic slices is advantageous compared to performing the analysis on real histological slices, since tomographic data can be recorded without destroying the sample and gives a fully unrestricted three-dimensional view inside the sample in addition to the stereological assessment.

The presented multimodal imaging method will be further improved through the development of a an automatic registration method for the merging of two multimodal datasets. Increasing the robustness of registration between the datasets will decrease the alignment work now needed prior to histological sectioning of the scanned sample. Currently, this alignment is the most crucial point in the whole workflow; with a more robust registration of the two-dimensional histological sections in the three-dimensional tomographic dataset we could increase the sample throughput.
\bibliographystyle{unsrtnat}
\bibliography{../../references}