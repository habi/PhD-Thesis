% !TEX root = ../Thesis.tex
\acresetall
\myChapter{Outlook}\label{ch:outlook}
\begin{flushright}{\slshape    
		So Long, and Thanks for All the Fish} \\ \medskip
    --- \defcitealias{Adams1984}{Douglas Adams}\citetalias{Adams1984} \citep{Adams1984}
\end{flushright}

\vfill

The tomographic images used in this work have served as a common ground truth to obtain insights into the mammalian lung. Morphometric parameters of the terminal airways obtained on three-dimensional reconstructions of tomographic data have been used to examine structural parameters of the gas-exchange region in rat lung samples.

The wide field scanning method presented in \autoref{ch:haberthuer2010} is used for routine measurements in our group. The current workflow uses custom MATLAB-scripts to merge the projections from the obtained subscans. The goal for the next few months is the integration of the wide field scanning method in the image processing pipeline~\cite{Hintermueller2010} of \acsu{tomcat}, especially into the software used to convert the projections to sinograms, which is dubbed \verb+Prj2Sin+. \verb+Prj2Sin+ already integrates the standard \SI{360}{\degree} off-center sample rotation method to double the horizontal field of view (as shown in \autoref{subfig:wfs360}). For such a \SI{360}{\degree} scan all projections are acquired in one sequence, and through relatively simple re-ordering of the projections on the disk \verb+Prj2Sin+ can compute the overlap between the images and calculate the merged sinograms. For a \ac{wf-srxtm} scan we generally acquire three separate subscans in three different directories. Stitching the projections directly with \verb+Prj2Sin+ instead of performing the overlap-search, reading and merging the original projections and writing the merged projections to disk prior to the generation of merged sinograms would greatly speed up the processing from acquired projections to merged sinograms to reconstructed tomographic wide field scan slices.

At the moment, the merging of one dataset with in total \num{10734} takes a bit more than one hour to perform with the MATLAB-scripts. Integrating the workflow into the image processing pipeline would both decrease the time used for merging these wide field scans by avoiding multiple reading and writing circles to disk and make the whole process available to other users of the beamline.

Nonetheless of these unresolved small issues, the feasibility of the enlargement of the horizontal field of view of tomographic endstations has been demonstrated and the resulting datasets have been used for the analysis of large volumes of the terminal airways with ultrahigh resolution. Future work will focus on the use of \ac{wf-srxtm} dataset for the extraction and analysis of the acini, as presented in \autoref{subsec:stereological analysis}. The presented method which uses so-called manhole covers to separate single acini from a central airway has been applied to multiple rat lung datasets recorded over the course of the postnatal development from day 4 to day 60. The definition of the location of the manhole covers in the airway tree is a process relying on manually locating the position of the junction in the airway based on morphological criteria like the changes in the airway wall and subsequent drawing of the manhole cover at these locations. Using the airway structure obtained from a prior skeletonization process as a guideline for the placements of these manhole covers is planned for the future. Such a guideline would both speed up the process of locating the exact positioning of the manhole covers and reduce the needed a priori morphological knowledge of the airway tree.

Using a combination of the manhole cover method for separating the acini with the skeletonization of the airway tree would make it possible to extract and segment large amounts of acini for the analysis in relatively short time. Currently achieved results hint towards a larger change of the acinar size during lung development than previously assumed\todo{or is it simply that the segmentation at D60 is not as easily possible? or do we really have much larger acini than at D36?}\todo{Citation about acinar size change?}.

Additionally, we plan to use the obtained \ac{wf-srxtm} datasets for the extraction of larger regions of the terminal airways than previously and currently available. These extracted regions will be used to improve currently available models of airflow inside the terminal airways \cite{Sznitman2007,Sznitman2009} \todo{What about ``Sznitman, Sutter, Altorfer, Tsuda, Stampanoni, Roesgen \& Schittny. Acinar flow simulations based on \threed terminal alveolar\ldots, submitted.''? $\rightarrow$ bug JCS!} using \ac{cfd}.

\section{Sebastien multimodal}
\begin{itemize}
	\item Automatic registration of multimodal datasets
	\item No longer relying on ``luck''
	\item Maybe even multimodal registration of \ac{em}-tomography
\end{itemize}

\section{Structural/Morphological Analysis of the lung}
\begin{itemize}
	\item Further work needs to be done with STEPanizer for the stereological analysis of the huge amount of scanned samples we now have\ldots
	\item Acinar size analysis of \ac{wf-srxtm} data
	\item Further proof of viability of tomographic imaging, not only \cite{Tsuda2008}, which has been taken up with mixed feelings\ldots
\end{itemize}

\section{Skeletonization}
\begin{itemize}
	\item ATS2009-Poster
	\item 2\textsuperscript{nd} year examination
	\item Collaboration with MeVis or even artorg.unibe.ch?
\end{itemize}

