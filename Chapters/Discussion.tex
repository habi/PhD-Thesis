% !TEX root = ../Thesis.tex
\acresetall
\myChapter{Discussion}\label{ch:discussion}
\begin{flushright}{\slshape And as soon as I'm done with these waffles, I shall discuss my evil  plan!} \\ \medskip
    --- \defcitealias{Zim}{Invader Zim}\citetalias{Zim} \citep{Zim}
\end{flushright}
\begin{flushright}{\slshape No discussion!} \\ \medskip
    --- Jean Reno as\defcitealias{Leon}{Léon}\citetalias{Leon} \citep{Leon}
\end{flushright}
\vspace{52mm}
The tomographic data obtained at the beamline for \ac{tomcat} offers an unmatched three-dimensional insight into the mammalian lung. Even if images of the terminal airway ends with higher resolution than the ones obtained at \ac{tomcat} have been obtained with \ac{em}\todo{citation}, it is impossible to visualize the larger three-dimensional structure of the terminal airway ends inside the lung using \ac{em} since the sample has to be sectioned prior to this method.

Tomography in contrast offers a non-destructive insight into the lung, tomographic microscopy using \ac{uct} enables the study of the major airways of swine~\cite{Litzlbauer2006}, rat \todo{citation} or mice lung~\cite{Langheinrich2004} and while \ac{srxtm} and \ac{wf-srxtm} enable the study of the functional lung units of the lung, the so-called acini with unmatched resolution and precision.

In this work three imaging methods based on ultrahigh resolution tomographic data of the terminal airway have been presented, the following sections discuss the major notable points of each of them.

\section{Multimodal Imaging}
The first method is a multimodal imaging method combining the advantages of tomographic imaging and \ac{em}. The tomographic data provides the unhindered three-dimensional information on the location of sub-micron particles in the terminal airway tree which is combined with the extremely high resolution of the detected particles for the precise analysis of those particles in the lung tissue.

Exposure to ultra-fine particles and nanoparticles produces by environmental or industrial sources is an important health problem. It is expected that inhaled particles are deposited in specific locations in the lung. With currently available methods, the exact location of these deposition sites cannot easily be analyzed. 

Classic histological sections analyzed using light and electron microscopy allow for a precise location of the inhaled particles in relation to the tissue, \ie make it possible to analyze the interaction of the particles with alveolar macrophages and epithelial cells~\cite{Muhlfeld2008}. But the precise localization of these interaction sites inside the airway tree are impossible to extract using histological sections since the three-dimensional structure of the sample is destroyed.

Registration of classic histological slices with the three-dimensional data obtained through \ac{srxtm} make it possible to localize sites of interaction along the airway tree inside the three-dimensional structure of the terminal airways. The method presented in chapter~\ref{ch:xrm2008} was only little more than a proof of concept in terms of the registration between the two imaging modalities. Careful alignment of the sample prior to physical sectioning made registration straightforward, since the images only needed to be corrected for rotation and translation to be able correlate between a physical slice imaged using \ac{em} and a virtual slice of the \ac{srxtm} dataset.

Figure~\ref{fig:correlation} shows the three-dimensional situation of different alignment situations of the two multimodal datasets. The situation shown in figure~\ref{subfig:correlation-planar} can be achieved through careful alignment of the sample prior to histological sectioning and corresponds to the situation presented in chapter~\ref{ch:xrm2008}. Figures~\ref{subfig:correlation-arbitrary1} and \ref{subfig:correlation-arbitrary3} depict situations where the two datasets are arbitrarily orientated with one or multiple degrees of rotation, respectively.

\renewcommand{\imsize}{\linewidth}
\begin{figure}[h]
	\centering
	\pgfmathsetlength{\imagewidth}{\imsize}%
	\pgfmathsetlength{\imagescale}{\imagewidth/1586}%
	\def\x{1160}
	\def\y{364}% scalebar-y at 90% of height of y=404px
	\subfloat[Planar orientation of slice]{%
		\begin{tikzpicture}[x=\imagescale,y=-\imagescale]
			\node[anchor=north west,inner sep=0pt,outer sep=0pt] at (0,0) {\includegraphics[width=\imagewidth]{img/discussion/R108C21Bt-mrg-planar}};
			% 515px = 4.3423mm > 100px = 844um > 59px = 500um, 12px = 100um
			%\draw[orange,|-|,thick] (537,266) -- (1043,172) node [sloped,midway,above] {\SI{4.3423}{\milli\meter} (2934px)};
			\draw[|-|,thick] (\x,\y) -- (\x+118,\y) node [right] {\SI{1}{\milli\meter}};
		\end{tikzpicture}%
		\label{subfig:correlation-planar}
		}\\%
	\subfloat[Arbitrary orientation of slice, one degree of freedom]{%
		\begin{tikzpicture}[x=\imagescale,y=-\imagescale]
			\node[anchor=north west,inner sep=0pt,outer sep=0pt] at (0,0) {\includegraphics[width=\imagewidth]{img/discussion/R108C21Bt-mrg-arbitrary1}};
			% 515px = 4.3423mm > 100px = 844um > 59px = 500um, 12px = 100um
			%\draw[white,|-|,thick] (537,266) -- (1043,172) node [sloped,midway,above] {\SI{4.3423}{\milli\meter} (2934px)};
			\draw[|-|,thick] (\x,\y) -- (\x+118,\y) node [right] {\SI{1}{\milli\meter}};
		\end{tikzpicture}%
		\label{subfig:correlation-arbitrary1}
		}\\%
	\subfloat[Arbitrary orientation of slice, multiple degrees of freedom]{%
		\begin{tikzpicture}[x=\imagescale,y=-\imagescale]
			\node[anchor=north west,inner sep=0pt,outer sep=0pt] at (0,0) {\includegraphics[width=\imagewidth]{img/discussion/R108C21Bt-mrg-arbitrary3}};
			% 515px = 4.3423mm > 100px = 844um > 59px = 500um, 12px = 100um
			%\draw[white,|-|,thick] (537,266) -- (1043,172) node [sloped,midway,above] {\SI{4.3423}{\milli\meter} (2934px)};
			\draw[|-|,thick] (\x,\y) -- (\x+118,\y) node [right] {\SI{1}{\milli\meter}};
		\end{tikzpicture}%
		\label{subfig:correlation-arbitrary3}
		}%
	\caption[Multimodal Imaging]{Multimodal Imaging: Alignment of datasets from two imaging modalities. Left: Sample from \ac{srxtm} with overlayed \ac{em} image (dark slice). Center: Five independent airway segments have been extracted and are three-dimensionally visualized with overlayed \ac{em}-image. Right: \ac{em}-image shown with three-dimensional orientation in relation to the \ac{srxtm} dataset. \subref{subfig:correlation-planar}: Planar orientation of the slice obtained with the second imaging modality. Through careful orientation of the sample prior to sectioning a registration is straightforward, since only the rotation in the plane of the slice has to be taken into account. \subref{subfig:correlation-arbitrary3}: Pitch-angle~\cite{YawPitchRoll} rotation between the two datasets. \subref{subfig:correlation-arbitrary3}: Pitch and roll rotation between the two datasets.}
	\label{fig:correlation}
	\todo[inline]{Slice is not multimodal but from \ac{srxtm}-set, so essentially I'm lying\dots $\rightarrow$ Do we need to show a LM or EM-slice or is this enough for concept?}
\end{figure}

Current work in our group focuses on an automatic registration method taking into account all six degrees of freedom \ie rotation and translation in x, y and z-plane. First results have already been presented as a Master Thesis by Sébastien \citet{Barre2009}.

\section{Finite Element}

\section{Widefield SRXTM}
\begin{itemize}
	\item tomography in general
	\item widefieldscan application
	\item skeletonization
	\item acinus detection \threed, marking in \twod makes it ``more understandable'' than on classic slices, see outlook in chapter~\ref{ch:outlook} and figure~\ref{fig:acinus overlay}.
\end{itemize}
\vspace{52mm}