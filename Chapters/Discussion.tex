% !TEX root = ../Thesis.tex
\acresetall
\myChapter{Discussion}\label{ch:discussion}
\begin{flushright}{\slshape And as soon as I'm done with these waffles, I shall discuss my evil  plan!} \\ \medskip
    --- \defcitealias{Zim}{Invader Zim}\citetalias{Zim} \citep{Zim}
\end{flushright}
\begin{flushright}{\slshape No discussion!} \\ \medskip
    --- Jean Reno as\defcitealias{Leon}{Léon}\citetalias{Leon} \citep{Leon}
\end{flushright}
\vspace{52mm}
The tomographic data obtained at the beamline for \ac{tomcat} offers an unmatched three-dimensional insight into the mammalian lung. Even if images of the terminal airway ends with higher resolution than the ones obtained at \ac{tomcat} have been obtained with electron microscopy\todo{citation}, it is impossible to visualize the larger three-dimensional structure of the terminal airway ends inside the l ung unsing electron microscopy since the sample has to be sectioned prior to this method.

Tomography in contrast offers a non-destructive insight into the lung, tomographic microscopy using \ac{uct} enables the study of the major airways of swine~\cite{Litzlbauer2006}, rat \todo{citation} or mice lung~\cite{Langheinrich2004} and while \ac{srxtm} and \ac{wf-srxtm} enable the study of the functional lung units of the lung, the so-called acini with unmatched resolution and precision.

In this work three imaging methods based on ultrahigh resolution tomographic data of the terminal airway have been presented, the following sections discuss the major notable points of each of them.

\section{Multimodal Imaging}
Tthe first method is a multimodal imgaging method combining the advantages of tomographic imaging and electron microscopy. The tomographic data provides the unhindered three-dimensional information on the location of submicron particles in the terminal airway tree which is combined with the extremely high resolution of the detected particles for the precise analysis of those particles in the lung tissue.

\section{Finite Element}

\section{Widefield SRXTM}
\begin{itemize}
	\item tomography in general
	\item widefieldscan application
	\item skeletonization
	\item acinus detection \threed, marking in \twod makes it ``more understandable'' than on classic slices, see outlook in chapter~\ref{ch:outlook} and figure~\ref{fig:acinus overlay}.
\end{itemize}
\vspace{52mm}