% !TEX root = ../Thesis.tex
\acresetall
\myChapter{Introduction}\label{ch:Introduction}
\begin{flushright}{\slshape    
		It's a magical world, Hobbes, ol' buddy\ldots\\
		\ldots Let's go exploring!}\\ \medskip
		--- Calvin, by\defcitealias{Watterson1996}{Bill Watterson}\citetalias{Watterson1996} \citep{Watterson1996}
\end{flushright}
\vspace{52mm}

Seeing and observing the object of interest has always been a crucial part of medical and biological research. If not by the naked eye, the object has been studied with ever-increasing resolution. In biomedical research, state of the art imaging methods include light and electron microscopy. Using advanced imaging methods like electron or x-ray microscopy, resolutions down to \SI{10}{\nano\meter} \todo{how much exactly?} can be reached, which enables an extremely accurate rendition of the sample to be examined .

\section{High resolution imaging}
One of the main problems with light and electron microscopy is the destructive sample preparation. For both methods, the sample has to be sectioned into physical slices with a thickness between 10--\SI{300}{\nano\meter}\todo{wie dünn genau? Citation?}. This process destroys the three-dimensional structure of the sample and makes is very time consuming to reconstruct the three-dimensional placement of the slices to extract the structural information. \citet{Woodward2005} have shown that a three-dimensional reconstruction of parts of the gas exchanging region of the avian lung is feasible, but the process is both extremely time-consuming and needs very precise registration of the stack of slices cut from the sample. An exception to the destructive sample preparation is offered by confocal microscopy \cite{Minsky1961}\todo{Other Citation than original patent?}. In confocal microscopy, light originating from outside the focal plane is excluded using a spatial pinhole. This allows for the definition of a section through a sample thicker than the optical focal plane of the objective. By positioning the focal plane stepwise through the whole sample, the three-dimensional information of the sample can be obtained. This method obviously only works with translucent samples.

In contrast to this, tomographic imaging makes it possible to non-destructively study the three-dimensional structure of a wide variety of samples. Two-dimensional projections of the sample can be taken fairly easily using a multitude of electromagnetic wavelengths (e.g.\ x-ray, light, Infrared). These two-dimensional images include partial three-dimensional information of the sample volume which has been transversed in the projection. If several projection images are acquired from different directions through the sample, the full three-dimensional information can be obtained. Using computed tomography~\cite{Hounsfield1976a}, a three-dimensional representation of the sample can be reconstructed and obtained. The theory and concepts behind the tomographic reconstruction are explained in more depth in chapter~\ref{ch:ct}.

\section{The mammalian lung}
\begin{itemize}
	\item Why the Mammalian lung?
	\item see chapter~\ref{ch:lung}
\end{itemize}

\section{Main goal of the Thesis}
From the start of my thesis at the Institute of Anatomy in September 2006 nearly everything I have worked on revolved around the tomographic non-destructive imaging and detection of the finest structures in the mammalian lung. Developing methods to analyze the ultrahigh resolution tomographic images obtained through the close collaboration with the team of the \ac{tomcat} beamline has been the main focus of my work. Pushing the boundaries of the available computing infrastructure in our group accompanied most projects, be it small or big.

Analyzing the lung with different methods\todo{xrm \& tsuda $\rightarrow$ skeleton, ATS2009-poster, etc.} led to...

While analyzing the structure of the airways during lung development using skeletonization algorithms we have seen that under certain circumstances the size of the functional lung unit---the so-called acinus---is bigger than the available field of view of the tomographic dataset resulting from a scan at \ac{tomcat}. More precisely, the acinus is 
\begin{enumerate}[i)]
	\item bigger than the field of view at the magnification needed to resolve the finest structures in the lung, the tissue septa between the alveoli in the terminal airway ends and
	\item seems to be growing over lung development.
\end{enumerate}

To solve point i) we needed a method to increase the field of view of \ac{tomcat} while keeping the resolution of the resulting tomographic images on the desired level. Such a method has been implemented at \ac{tomcat} and published (see chapter~\ref{ch:haberthuer2010}) for others to implement at other tomographic beamlines.

Point ii) is ongoing research, chapter~\ref{ch:outlook} aims to present an outlook based on the results presented in this thesis. 

\section{Outline of the Thesis}
This thesis is structured into the following parts:
\begin{itemize}
	\item [Part \ref{part:introduction}] contains this introduction and gives a short overview over the lung development in chapter~\ref{ch:lung}. Chapter~\ref{ch:ct} explains the most important concepts in computed tomography and gives a short description of the \acf{tomcat} beamline, where all high resolution tomography experiments of this work have been performed.
	\item [Part \ref{part:results}] contains publications written during the time of my Ph.D.\ Thesis at the Institute of Anatomy.
	\begin{itemize}
		\item [Chapter~\ref{ch:XRM2008}] consists of a proceeding written for the 9\textsuperscript{th} \href{http://xrm2008.web.psi.ch/}{International Conference on X-Ray Microscopy} in Zürich, Switzerland, from the 21\textsuperscript{st} to the 25\textsuperscript{th} July 2008. This proceeding describes a method to precisely localise instilled sub-micron sized gold particles in rat lungs through a combination of high resolution synchrotron radiation tomographic microscopy and \acl{tem} and has been published in the Conference Series of the \href{http://iopscience.iop.org/1742-6596/}{Journal of Physics}
		\item [Chapter~\ref{ch:Tsuda2008}] consists of a paper published to the \href{http://jap.physiology.org/}{Journal of Applied Physiology} in which a new image processing method for accurate three-dimensional reconstruction of the gas exchange region of the lung is presented. This method is based on a finite element analysis of datasets obtained with high resolution synchrotron radiation tomographic microscopy.
		\item [Chapter~\ref{ch:Haberthuer2010}] presents a method to increase the field of view of synchrotron radiation tomographic microscopy while keeping the advantage of a high resolution, something which is generally not possible with such an increase in the field of view. Additionally---through optimization of the acquisition protocol---the radiation dose inflicted on the sample can be greatly reduced. This paper has been published in the \href{http://journals.iucr.org/s/}{Journal of Synchrotron Radiation}.
	\end{itemize}
	\item [Part \ref{part:discussion}] contains two chapters with a closing discussion of the obtained results and an outlook on further work.
	\item [Part \ref{part:back matter}] finishes this thesis with the necessary documents for the \href{http://www.gcb.unibe.ch}{Graduate School for Cellular and Biomedical Sciences}, the bibliography and an index\todo{do we have an index?}.
\end{itemize}