% !TEX root = ../Thesis.tex
\acresetall
\myChapter{Publications}\label{ch:publications}
\begin{flushright}{\slshape I'm throwing rocks tonight. Mark it, Dude.} \\ \medskip
	--- Steve Buscemi\\as Donny in\defcitealias{TheBigLebowski}{The Big Lebowski}\citetalias{TheBigLebowski} \citep{TheBigLebowski}
\end{flushright}
\vspace{6cm}

The three chapters~\ref{ch:xrm2008}, \ref{ch:tsuda2008} and \ref{ch:haberthuer2010} of this document are copies of publications I have authored or collaborated on. They represent the published results achieved during the time frame of this thesis.

\section{\autoref{ch:xrm2008} -- Multimodal imaging}
\autoref{ch:xrm2008} contains a proceeding that has been published in the Conference Series of the \href{http://iopscience.iop.org/1742-6596/}{Journal of Physics} after the 9\textsuperscript{th} \href{http://xrm2008.web.psi.ch/}{International Conference on X-Ray Microscopy} in Zürich, Switzerland. The manuscript presents a method for localization and three-dimensional imaging of sub-micron gold particles in rat lungs. Two batches of gold particles with sizes of \SI{200}{\nano\meter} and \SI{700}{\nano\meter} have been applied to rat lung samples through intratracheal instillation. The samples have been scanned with high resolution \ac{srxtm} at the \ac{tomcat} beamline of the \ac{sls} with a voxel size of \SI{350}{\nano\meter}. This resolution was sufficient to detect the exact three-dimensional localization of single and clustered gold particles with a size of \SI{700}{\nano\meter} in alveoli, alveolar ducts and terminal bronchioli in the rat lung samples. 

As the particles with a diameter of \SI{200}{\nano\meter} were below the detection limit of \ac{tomcat}, we were only able to accurately localize them by \ac{tem}. For this, the samples was sectioned into \SI{300}{\nano\meter} thick sections and analyzed using \ac{tem}. We were able to verify the exact locations of the gold particles of both sizes in the lung tissue. Gold particles with a size of \SI{700}{\nano\meter} were localized in consecutive sections even if they were not directly visible on certain slices. Since these particles were pulled out of the resin, we sometimes only observed a hole were gold particles would have been expected.

The high correlation between the virtual sections of the tomographic dataset and the real sections imaged through \ac{tem} makes the combination of both imaging modalities very well suited to obtain the full unrestricted 3D access using \ac{srxtm} and to verify the localization of the particles in the 3D-space with very high resolution using \ac{tem}.

\section{\autoref{ch:tsuda2008} -- \acs{fe} \threed reconstruction of the acinus}
\autoref{ch:tsuda2008} presents a new image processing method for accurate three-dimensional reconstruction of the gas exchange region of the lung. The method has been published in the \href{http://jap.physiology.org/}{Journal of Applied Physiology} and is based on a \ac{fe} analysis of the tomographic datasets recorded at \ac{tomcat}. 

The datasets have been segmented using an iteratively determined threshold. This threshold was chosen to match computed morphological parameters of the resulting three-dimensional reconstruction with the stereologically\graffito{Stereology refers to the mathematical methods for defining physical properties of irregular three-dimensional structures using two-dimensional sections obtained by physical or optical imaging techniques~\cite{Hsia2010}.} determined real morphological parameters of samples from the same lungs.

In this work we have been able to highlight numerous advantages of a three-dimensional analysis of the reconstructions over traditional stereology. One of the most fundamental advantages of the tomographic analysis is the non-destructive method of obtaining the informations. Using traditional methods, the sample has to be cut into histological sections, which essentially destroys the integrity of the sample. Using the tomographic dataset, a thorough analysis of the inner structure of sample can also be carried out on slices, but these slices are virtual. Another advantage of the virtual slicing through the sample is that slices with arbitrary orientation can be generated throughout the sample as long as needed, and interesting regions can be iteratively improved.

\section{\autoref{ch:haberthuer2010} -- Expansion of the field of view}
In \autoref{ch:haberthuer2010} I present the so-called \emph{wide field scanning} method developed in order to cover a larger field of view at \ac{tomcat}. This larger field of view was needed to enable the three-dimensional analysis of the functional lung unit, the so-called acinus which seems to be growing to a larger extent than previously described \cite{Massaro1985,Massaro1992}.

Additionally, since the smallest structures we need to unambiguously detect are the alveolar septa with a thickness of several micrometers, we need to obtain tomographic scans with large magnifications, \ie high resolution in the order of one micron. 

Generally, those two points can not be satisfied with a single tomographic scan, since an increase in the field of view can only be acquired with low magnification and vice-versa. The method presented in this chapter overcomes this limitation through merging of several high resolution scans to one scan overlapping a large field of view. Additionally---through optimization of the acquisition protocol---the scanning time is minimized, which greatly reduces the radiation dose inflicted on the sample, which is a first step towards in-vivo \ac{srxtm} at \ac{tomcat}.

This so-called \ac{wf-srxtm} was implemented as a proof of concept during my post-graduate master thesis of the \href{http://www.biomed.ee.ethz.ch/nds/}{Master of advanced Studies ETH in Medical Physics} in the \ac{tomcat} group. In the following months, the method has been refined into a scanning protocol and is now routinely applied to obtain tomographic datasets of rat lung samples with both large field of view and high resolution. This paper is currently in press and will be published in the \href{http://journals.iucr.org/s/}{Journal of Synchrotron Radiation}.

% \renewcommand\bibname{Bibliography | Publications}%
% \bibliographystyle{unsrtnat}%
% \bibliography{../../references}%
% \addcontentsline{toc}{section}{\bibname}%