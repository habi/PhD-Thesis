% !TEX root = ../Thesis.tex
\acresetall
\pdfbookmark[1]{Publications}{publications}
\myChapter{Publications}\label{ch:publications}
\begin{flushright}{\slshape I'm throwing rocks tonight. Mark it, Dude.} \\ \medskip
    --- Steve Buscemi as Donny in\defcitealias{TheBigLebowski}{The Big Lebowski}\citetalias{TheBigLebowski} \citep{TheBigLebowski}
\end{flushright}
\vspace{52mm}

The three chapters \ref{ch:xrm2008}, \ref{ch:tsuda2008} and \ref{ch:haberthuer2010} of this document are verbatim copies of publications I have authored or collaborated on. They represent the published results achieved during the time frame of this thesis.

In chapter~\ref{ch:XRM2008} contains a proceeding that has been published in the Conference Series of the \href{http://iopscience.iop.org/1742-6596/}{Journal of Physics} after the 9\textsuperscript{th} \href{http://xrm2008.web.psi.ch/}{International Conference on X-Ray Microscopy} in Zürich, Switzerland. In this manuscript we presented a method for localization and three-dimensional imaging of sub-micron sized gold particles in rat lungs. Two batches of gold particles with sizes of \SI{200}{\nano\meter} and \SI{700}{\nano\meter} have been applied to rat lung samples through intratracheal instillation. The samples have been scanned with high resolution \ac{srxtm} at the \ac{tomcat} beamline of the \ac{sls} with a voxel size of \SI{350}{\nano\meter}. This resolution was sufficient to detect the exact three-dimensional localization of single and clustered gold particles with a size of \SI{700}{\nano\meter} in alveoli, alveolar ducts and terminal bronchioli in the rat lung samples.

After scanning the samples at \ac{tomcat} we analyzed physical sections using \ac{tem} to verify the locations of the gold particles using a commonly accepted method. We were expecting to be able to correlate between both the virtual sections of the tomographic dataset and the real sections imaged through \ac{tem}, but were surprised how accurately we were able to correlate between the two imaging modalities. As expected, the particles with a diameter of \SI{200}{\nano\meter} were smaller than the detection capabilities of \ac{tomcat}, so we were only able to localize them using \ac{tem}.

The combination of both imaging modalities made it possible to obtain the full
unrestricted 3D access using \ac{srxtm} and to verify the localization of the particles in the 3D-space with very high resolution using \ac{tem}.

Chapter~\ref{ch:Tsuda2008} presents a new image processing method for accurate three-dimensional reconstruction of the gas exchange region of the lung. The method has been  published in the \href{http://jap.physiology.org/}{Journal of Applied Physiology} and is based on a \ac{fe} analysis of the tomographic datasets recorded at \ac{tomcat}. 

The datasets have been segmented using an iteratively determined threshold, which was chosen to match morphological parameters of the resulting three-dimensional reconstruction with the same morphological parameters of the lung samples. The structural parameters of the lung samples have been previously determined using classical stereological\graffito{Stereology is used for extracting quantitative information about a three-dimensional sample from measurements made on two-dimensional sections of the material}\todo{other explanation of ``stereology'' in one sentence. with citation?} methods.

In this work we have been able to highlight numerous advantages of a three-dimensional analysis of the reconstructions over traditional stereology. One of the most fundamental advantages of the tomographic analysis is the non-destructive method of obtaining the informations. Using traditional methods, the sample has to be sectioned, which essentially destroys the integrity of the sample. Using the tomographic dataset, a thorough analysis of the inner structure of sample can also be carried out on slices, but these slices are virtual. These virtual slices through the dataset also offers another advantage; slices can be generates in arbitrary directions through the sample as long as needed, and interesting regions can be iteratively improved. See the discussion in chapter~\ref{ch:discussion} for additional advantages\todo{acinus detection \threed, marking in \twod makes it ``more understandable'' than on classic slices.}.

Chapter~\ref{ch:Haberthuer2010} presents a method to increase the field of view of synchrotron radiation tomographic microscopy while keeping the advantage of a high resolution, something which is generally not possible with such an increase in the field of view. The work presented in this chapter started out of a need in our group. We noticed that the functional lung unit---the so-called acinus---seems to be growing to a larger extend than previously described \todo{citation?} and needed a method to easily cover a larger field of view with the tomographic reconstructions of our samples obtained at \ac{tomcat}.

Since the smallest structures we need to unambiguously detect are the alveolar septa with a thickness of several micrometers, we need to obtain tomographic scans with large magnifications, \ie high resolution in the order of one micron. Additionally, we strive for a large field of view to visualize an entire acinus. Usually, a large field of view resulting in a large sample volume can only be acquired with low magnification and vice-versa. To overcome this limitation the method presented in this chapter was implemented as a proof of concept during my master thesis of the \href{http://www.biomed.ee.ethz.ch/nds/}{MAS ETH in Medical Physics} at the \ac{psi} \cite{Haberthuer2008c}. In the following months, the method has been refined into a scanning method at \ac{tomcat} and is now routinely applied to obtain tomographic datasets of rat lung samples with both large field of view and high resolution.

Additionally---through optimization of the acquisition protocol---the radiation dose inflicted on the sample can be greatly reduced, which is a first step towards in-vivo \ac{srxtm} at \ac{tomcat}. This paper has been published in the \href{http://journals.iucr.org/s/}{Journal of Synchrotron Radiation}.