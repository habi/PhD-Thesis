% !TEX root = ../Thesis.tex
\myChapter{The mammalian lung}\label{ch:lung}
\begin{flushright}{\slshape I'm throwing rocks tonight. Mark it, Dude.} \\ \medskip
    --- \defcitealias{TheBigLebowski}{Steve Buscemi as Donny}\citetalias{TheBigLebowski} \citep{TheBigLebowski}
\end{flushright}
\bigskip
\section{Interestingness}

\section{Lung Development}
During lung development all the internal structures like the conducting airways, the blood vessel network and the gas exchange area are formed. The lung is specifically designed to provide this large gas exchange surface\todo{Tennisplatz -> Weibel} where capillary blood efficiently gets in close contact to the air inside the lung structure. 

Mammalian lung development can be divided into five overlapping stages~\cite{Schittny2004}\todo{Do we have a Graph for this?}: Organogenesis starts with a outpouching of the foregut resulting in the appearance of the lung buds. Subsequently, the conducting and parts of the respiratory airways are formed by a successive cycle of branching and growth starting at the lung buds, which is called branching morphogenesis. Most of this process takes place during the pseudoglandular stage.

This stage is followed by intermediate stages, which are called canalicular and saccular. During the canalicular stage a first functional gas exchange surface---the air-blood barrier---is formed. The saccular stage marks the switch from branching to septation morphogenesis.

During the alveolar stage the distal part of the bronchial tree is enlarged by a lifting off of additional septa from already existing septa (septation/alveolarization). 

In order to optimize gas exchange after bulk alveolarization is completed, the interalveolar septa and their capillary networks are remodeled during the phase of microvascular maturation. At this point lung development is considered as being finished and normal growth of the organ follows.

The time point of birth differs between mammals, relative to the state of lung development. In humans, birth happens at the beginning of the alveolar stage.