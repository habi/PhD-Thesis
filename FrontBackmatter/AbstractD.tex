% !TEX root = ../Thesis.tex
\acresetall
\pdfbookmark[1]{Zusammenfassung}{Zusammenfassung}%
\hfill
\vfill
\begingroup%
\let\clearpage\relax%
\let\cleardoublepage\relax%
\let\cleardoublepage\relax%
\selectlanguage{ngerman}%
\chapter*{Zusammenfassung}
Um die Lungenentwicklung und die strukturellen Parameter der terminalen Luftwege detailliert analysieren zu können, muss ein dreidimensionales Abbildungsverfahren verwendet werden, denn auf klassischen histologischen Schnitten des Lungengewebes können nicht alle strukturellen Parameter erfasst werden.

Mit tomographischer Mikroskopie basierend auf Synchrotronstrahlung können solche dreidimensionalen volumetrische Informationen von praktisch beliebigen Proben gewonnen werden. Die tomographische Mikroskopie bietet verschiedene Vorzüge, welche sie zu einer bestens geeigneten Methode machen, um Lungenproben minuziös in drei Dimensionen zu untersuchen; namentlich können die Proben zerstörungsfrei untersucht werden sowie bieten die resultierenden Daten höchste Auflösungen im Mikrometerbereich und können in kürzester Zeit, normalerweise innerhalb weniger Minuten aufgenommen werden.

Die vorliegende Arbeit präsentiert mehrere Methoden, um die terminalen Luftwege der Lunge mittels höchstaufgelösten tomographischen Daten zu untersuchen. Durch eine Kombination dieser Daten mit Elektronenmikroskopiebildern konnten wir die genauen Orte der Ablagerung von Partikeln kleiner als ein Mikrometer in den terminalen Luftwegen lokalisieren und die genauen Eigenschaften der Partikel untersuchen.

Weiter haben wir dreidimensionale Rekonstruktionen von tomographischen Daten benutzt, um die struturellen Parameter der terminalen Luftwege zu untersuchen. Der Vergleich der erlangten Resultate mit Resultaten, welche aufgrund von klassischen Histologieschnitten erhoben wurden zeigt, dass diese strukturellen Parameter genausogut auf tomographischen Daten erhoben werden können, ohne die Genauigkeit der Resultate einzuschränken. Dies hebt den Vorteil der zerstörungsfreien tomographischen Abbildungsmethode hervor.

Die dritte Methode, welche in dieser Arbeit präsentiert wird, ermöglicht es, tomographische Datensätze von grossen\graffito{\emph{Gross} heisst in diesem Zusammenhang mehrere Kubikmillimeter.} Lungenvolumina in höchster Auflösung aufzunehmen. So aufgenommene Datensätze ermöglichen es, Strukturänderungen der terminalen Luftwege in der Säugetierlunge während der postnatalen Lungenentwicklung zu untersuchen. Das vorgestellte Verfahren stellt einen Durchbruch in der tomographischen Mikroskopie dar, denn bis jetzt ging eine Vergrösserung der abgebildeten Volumina immer mit einer Verkleinerung der Auflösung der erzielten Daten einher. Dieser Nachteil konnte mit der neu entwickelten, sogenannten \emph{wide field scanning} Methode überwunden werden.%\graffito{Mami, chunnsch no drus?}
\selectlanguage{english}%
\endgroup%