% !TEX root = ../Thesis.tex
%*******************************************************
% Abstract
%*******************************************************
%\renewcommand{\abstractname}{Abstract}
\pdfbookmark[1]{Abstract}{Abstract}%
\begingroup%
\let\clearpage\relax%
\let\cleardoublepage\relax%
\let\cleardoublepage\relax%
\selectlanguage{ngerman}%
\pdfbookmark[1]{Zusammenfassung}{Zusammenfassung}%
\chapter*{Zusammenfassung}
Um die Lungenentwicklung, im speziellen die strukturellen Parameter der terminalen Luftwege detailliert analysieren zu können, muss ein dreidimensionales Abbildungsverfahren genutzt werden. Mit tomographische Röntgenmikroskopie basierend auf Synchrotronstrahlung können solche dreidimensionalen volumetrische Informationen von praktisch beliebigen Proben gewonnen werden. Die tomographische Mikroskopie bietet verschiedene Vorzüge, die sie zu einer bestens geeigneten Methode machen um Lungenproben minuziös in drei Dimensionen zu untersuchen; namentlich können die Proben zerstörungsfrei untersucht werden sowie bieten die resultierenden Daten höchste Auflösungen im Mikrometerbereich und können in wenigen Minuten aufgenommen werden.

Die vorliegende Arbeit präsentiert mehrere Methoden, um die terminalen Luftwege der Lunge mittels höchstaufgelösten tomographischen Daten zu untersuchen. Durch eine Kombination dieser Daten mit Elektronenmikroskopiebildern haben wir Partikel, welche kleiner als ein Mikrometer waren in den terminalen Luftwegen lokalisiert und konnten ihre Eigenschaften genau untersuchen.

Weiter haben wir dreidimensionale Rekonstruktionen von tomographischen Daten benutzt, um die struturellen Parameter der terminalen Luftwege zu untersuchen. Die Vergleich der erlangten Resultate mit Resultaten von klassischen Histologieschnitten zeigt, dass diese strukturellen Parameter genausogut auf tomographischen Daten erhoben werden können, ohne die Genauigkeit einzuschränken. Dies hebt den Vorteil der zerstörungsfreien tomographischen Abbildungsmethode hervor.

Die dritte Methode, welche in dieser Arbeit präsentiert wird, ermöglicht es, tomographische Datensätze von grossen\graffito{Gross heisst in diesem Zusammenhang mehrere Kubikmillimeter.} Lungenvolumina in höchster Auflösung aufzunehmen. So aufgenommene Datensätze ermöglichen es, Strukturänderungen der terminalen Luftwege in der Säugetierlunge zu untersuchen. Das vorgestellte Verfahren stellt einen Durchbruch in der tomographischen Mikroskopie dar, denn bis jetzt ging eine Vergrösserung der abgebildeten Volumina immer mit einer Verkleinerung der Auflösung der erzielten Daten einher\graffito{Mami, chunnsch no drus?}.
\selectlanguage{english}%
\endgroup%