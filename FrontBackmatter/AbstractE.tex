% !TEX root = ../Thesis.tex
\acresetall
\pdfbookmark[1]{Abstract}{Abstract}%
\hfill
\vfill
\begingroup%
\let\clearpage\relax%
\let\cleardoublepage\relax%
\let\cleardoublepage\relax%
\chapter*{Abstract}%
Studying lung development in all its characteristics, especially analyzing the structural parameters of the terminal airways needs a fully three-dimensional imaging modality, since not all structural parameters can be assessed on classic histological sections.

Using synchrotron radiation based tomographic microscopy three-dimensional volumetric information of arbitrary samples can be obtained. The distinct advantages offered by tomographic microscopy make it an extremely well suited imaging method for the minute analysis of lung samples in three dimensions. Tomographic microscopy permits to study the samples in a non-destructive way and allows the acquisition of tomographic datasets with ultra high resolution in the micrometer scale in a short time, usually within a few minutes.

This work presents several methods for the analysis of the terminal airway using high resolution tomographic datasets. Using a combination of this imaging modality with transmission electron microscopy data we localized the deposition sites of sub-micrometer sized particles in the terminal airways and studied their properties.

Three-dimensional tomographic reconstructions were used for the analysis of structural parameters of the terminal airways. The achieved results confirmed the usability of tomographic data for confirming structural parameters obtained from classic histological slices, which highlights the advantage of the non-destructive scanning method.

The third method presented in this work makes it possible to record tomographic dataset of large\graffito{In the context of this work \emph{large} means with a volume of several cubic millimeters.} sample volumes with ultra high resolution. Using such datasets, we analyzed structural changes of the terminal airways of the mammalian lung during postnatal lung development. The presented method achieves a breakthrough for tomographic imaging, since generally a large field of view had to be traded for a high resolution, which is no longer necessary with the application of the so-called wide field scanning.
\endgroup%