% !TEX root = ../Thesis.tex
%*******************************************************
% Abstract
%*******************************************************
%\renewcommand{\abstractname}{Abstract}
\pdfbookmark[1]{Abstract}{Abstract}%
\begingroup%
\let\clearpage\relax%
\let\cleardoublepage\relax%
\let\cleardoublepage\relax%
\chapter*{Abstract}%
Studying lung development in all its characteristics, especially analyzing the structural parameters of the terminal airway ends needs a fully three-dimensional imaging modality. Using synchrotron radiation based tomographic microscopy such three-dimensional volumetric information of arbitrary sample can be obtained. Tomographic microscopy has distinct advantages which make it an extremely well suited imaging method for the minute analysis of lung samples in three dimensions, namely that it permits to study the samples in a non-destructive way and allows the acquisition of tomographic datasets with ultra high resolution in the micrometer scale and within a few minutes.

This work presents several methods for the analysis of the terminal airway using high resolution tomographic datasets. Using a combination of this imaging modality with transmission electron microscopy data we localized sub-micrometer sized particles in the terminal airways and studied their properties.

Three-dimensional tomographic reconstructions were used for the analysis of structural parameters of the terminal airways. The achieved results confirmed the usability of tomographic data for confirming structural parameters obtained from classic histological slices, which highlights the advantage of the non-destructive scanning method.

The third method presented in this work makes it possible to record tomographic dataset of large\graffito{Large in this context means with a volume of several cubic millimeters.} sample volumes with ultra high resolution. Using such datasets, structural changes of the terminal airways of the mammalian lung can be studied. The presented method achieves a breakthrough for tomographic imaging, since generally a large field of view had to be traded for a high resolution.
\endgroup%