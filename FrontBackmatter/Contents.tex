% !TEX root = ../Thesis.tex
%*******************************************************
% Table of Contents
%*******************************************************
%\phantomsection
\refstepcounter{dummy}
\pdfbookmark[1]{\contentsname}{tableofcontents}
\setcounter{tocdepth}{2} % <-- 2 includes up to subsections in the ToC
\setcounter{secnumdepth}{3} % <-- 3 numbers up to subsubsections
\manualmark
\markboth{\spacedlowsmallcaps{\contentsname}}{\spacedlowsmallcaps{\contentsname}}
\tableofcontents 
\automark[section]{chapter}
\renewcommand{\chaptermark}[1]{\markboth{\spacedlowsmallcaps{#1}}{\spacedlowsmallcaps{#1}}}
\renewcommand{\sectionmark}[1]{\markright{\thesection\enspace\spacedlowsmallcaps{#1}}}
%*******************************************************
% List of Figures and of the Tables
%*******************************************************
\clearpage

\begingroup 
    \let\clearpage\relax
    \let\cleardoublepage\relax
    \let\cleardoublepage\relax
    %*******************************************************
    % List of Figures
    %*******************************************************    
    %\phantomsection 
    \refstepcounter{dummy}
    %\addcontentsline{toc}{chapter}{\listfigurename}
    \pdfbookmark[1]{\listfigurename}{lof}
    \listoffigures

    \vspace*{8ex}

    %*******************************************************
    % List of Tables
    %*******************************************************
    %\phantomsection 
    \refstepcounter{dummy}
    %\addcontentsline{toc}{chapter}{\listtablename}
    \pdfbookmark[1]{\listtablename}{lot}
    \listoftables
        
    \vspace*{8ex}
%   \newpage
    
	%*******************************************************
	% List of Listings
	%*******************************************************      
	%\phantomsection 
	\refstepcounter{dummy}
    %\addcontentsline{toc}{chapter}{\lstlistlistingname}
    \pdfbookmark[1]{\lstlistlistingname}{lol}
    \lstlistoflistings 

    \vspace*{8ex}

    %*******************************************************
    % Acronyms
    %*******************************************************
    %\phantomsection 
    \refstepcounter{dummy}
    \pdfbookmark[1]{Acronyms}{acronyms}
    \markboth{\spacedlowsmallcaps{Acronyms}}{\spacedlowsmallcaps{Acronyms}}
    \chapter*{Acronyms}
	\begin{acronym}[WF-SRXTM]
		\acro{cad}[CAD]{Computer Aided Design}
		\acro{ccd}[CCD]{charge-coupled device}
		\acro{cfd}[CFD]{computational fluid dynamics}
		\acro{ct}[CT]{Computed tomography}
		\acro{cpu}[CPU]{Central processing unit}
		\acro{epics}[EPICS]{Experimental Physics and Industrial Control System}: Control system by the national laboratory of \href{http://www.aps.anl.gov/epics/}{Argonne in the USA}, used to control \acs{tomcat}.
		\acro{em}[EM]{electron microscopy}
		\acro{fe}[FE]{finite element}
		\acro{fft}[FFT]{Fast Fourier Transform}
		\acro{gbha}[GBHA]{grid-based hexahedral algorithm}
		\acro{istb}[ISTB]{Institute for Surgical Technology \& Biomechanics}
		\acro{lst}[LST]{Laplacian smoothing technique}
		\acro{uct}[\micro CT]{micro-computed tomography}
		\acro{mri}[MRI]{Magnetic resonance imaging}
		\acro{oca}[OCA]{object connectivity analysis}
		\acro{oln}[OLN]{object label number}
		\acro{pc}[PC]{Personal computer}
		\acro{psi}[PSI]{Paul Scherrer Institut}: The \acs{psi} is the largest research centre for natural and engineering sciences in Switzerland.
		\acro{ram}[RAM]{Random-access memory}
		\acro{roi}[ROI]{Region of interest}
		\acro{se}[SE]{Standard error}: Standard deviation of a sample normalized by the square root of the sample size.
		\acro{tem}[TEM]{transmission electron microscopy}
		\acro{tomcat}[TOMCAT]{TOmographic Microscopy and Coherent rAdiology experimenTs}: \acs{tomcat} is the beamline at the \acs{sls}, where all experiments of this thesis have been performed.
		\acro{sls}[SLS]{Swiss Light Source}: The \acs{sls} at the \acs{psi} is a third-generation synchrotron light source.
		\acro{snr}[SNR]{signal-to-noise ratio}
		\acro{sr}[SR]{Synchrotron radiation}
		\acro{srct}[SRCT]{synchrotron radiation computed tomography}
		\acro{srxtm}[SRXTM]{synchrotron radiation based x-ray tomographic microscopy}
		\acro{stl}[STL]{Surface Tessellation or Standard Triangulation Language or Standard Tessellation Language}: A file format used as a Quasi-Standard of many \acs{cad}-systems.
		\acro{vrml}[VRML]{Virtual Reality Modeling Language}
		\acro{wf-srxtm}[WF-SRXTM]{wide field synchrotron radiation based x-ray tomographic microscopy}: The method to increase the field of view of tomographic imaging stations presented in \autoref{ch:haberthuer2010}.
		\acro{xml}[XML]{Extensible Markup Language}: A set of rules for encoding documents electronically.
		\acro{yag}[YAG]{Yttrium aluminium garnet}
	\end{acronym}
\endgroup

\cleardoublepage