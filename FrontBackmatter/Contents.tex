% !TEX root = ../Thesis.tex
%*******************************************************
% Table of Contents
%*******************************************************
%\phantomsection
\refstepcounter{dummy}
\pdfbookmark[1]{\contentsname}{tableofcontents}
\setcounter{tocdepth}{2} % <-- 2 includes up to subsections in the ToC
\setcounter{secnumdepth}{3} % <-- 3 numbers up to subsubsections
\manualmark
\markboth{\spacedlowsmallcaps{\contentsname}}{\spacedlowsmallcaps{\contentsname}}
\tableofcontents 
\automark[section]{chapter}
\renewcommand{\chaptermark}[1]{\markboth{\spacedlowsmallcaps{#1}}{\spacedlowsmallcaps{#1}}}
\renewcommand{\sectionmark}[1]{\markright{\thesection\enspace\spacedlowsmallcaps{#1}}}
%*******************************************************
% List of Figures and of the Tables
%*******************************************************
\clearpage

\begingroup 
    \let\clearpage\relax
    \let\cleardoublepage\relax
    \let\cleardoublepage\relax
    %*******************************************************
    % List of Figures
    %*******************************************************    
    %\phantomsection 
    \refstepcounter{dummy}
    %\addcontentsline{toc}{chapter}{\listfigurename}
    \pdfbookmark[1]{\listfigurename}{lof}
    \listoffigures

    \vspace*{8ex}

    %*******************************************************
    % List of Tables
    %*******************************************************
    %\phantomsection 
    \refstepcounter{dummy}
    %\addcontentsline{toc}{chapter}{\listtablename}
    \pdfbookmark[1]{\listtablename}{lot}
    \listoftables
        
    \vspace*{8ex}
	%\newpage
    
	%*******************************************************
	% List of Listings
	%*******************************************************      
	%\phantomsection 
%	\refstepcounter{dummy}
    %\addcontentsline{toc}{chapter}{\lstlistlistingname}
%    \pdfbookmark[1]{\lstlistlistingname}{lol}
%    \lstlistoflistings 
%    \vspace*{8ex}
	\newpage

    %*******************************************************
    % Acronyms
    %*******************************************************
    %\phantomsection
    \refstepcounter{dummy}
    \pdfbookmark[1]{Acronyms}{acronyms}
    \markboth{\spacedlowsmallcaps{Acronyms}}{\spacedlowsmallcaps{Acronyms}}
    \chapter*{Acronyms}
	\begin{acronym}[WF-SRXTM]
		\acro{avv}[AVV]{I have no idea what an \acsu{avv} is, but it's used in a Monty Python sketch\ldots}
		\acro{bp}[BP]{Blood Pressure}
		\acro{cad}[CAD]{Computer Aided Design}
		\acro{ccd}[CCD]{Charge-Coupled Device}
		\acro{cfd}[CFD]{Computational Fluid Dynamics}
		\acro{ct}[CT]{Computed Tomography}
		\acro{cpu}[CPU]{Central Processing Unit}
		\acro{eeg}[EEG]{Electroencephalography}: Using an \acsu{eeg}, the electrical activity of the brain can be measured.
		\acro{epics}[EPICS]{Experimental Physics and Industrial Control System}: Control system by the national laboratory of \href{http://www.aps.anl.gov/epics/}{Argonne in the USA}, used to control \acs{tomcat}.
		\acro{em}[EM]{Electron Microscopy}: An electron microscope produces an electronically magnified image of a specimen for detailed observation. The electron microscope uses an electron beam to illuminate the specimen and create a magnified image of it.
		\acro{fe}[FE]{Finite Element}
		\acro{fft}[FFT]{Fast Fourier Transform}
		\acro{gbha}[GBHA]{Grid-Based Hexahedral Algorithm}
		\acro{istb}[ISTB]{Institute for Surgical Technology \& Biomechanics}
		\acro{lst}[LST]{Laplacian Smoothing Technique}
		\acro{uct}[\micro CT]{micro-Computed Tomography}
		\acro{mri}[MRI]{Magnetic Resonance Imaging}
		\acro{oca}[OCA]{Object Connectivity Analysis}
		\acro{oln}[OLN]{Object Label Number}
		\acro{pc}[PC]{Personal Computer}
		\acro{psi}[PSI]{Paul Scherrer Institut}: The \acs{psi} is the largest research centre for natural and engineering sciences in Switzerland.
		\acro{ram}[RAM]{Random-Access Memory}
		\acro{roi}[ROI]{Region of Interest}
		\acro{se}[SE]{Standard Error}: Standard deviation of a sample normalized by the square root of the sample size.
		\acro{tem}[TEM]{Transmission Electron Microscopy}: \acs{tem} is a microscopy  technique whereby a beam of electrons is transmitted through an very thin specimen.
		\acro{tomcat}[TOMCAT]{TOmographic Microscopy and Coherent rAdiology experimenTs}: \acs{tomcat} is the beamline at the \acs{sls}, where all experiments of this thesis have been performed.
		\acro{sem}[SEM]{Scanning Electron Microscopy}: \acs{sem} is an electron microscopy technique that images the sample surface by scanning it with a high-energy beam of electrons in a raster scan pattern.
		\acro{sls}[SLS]{Swiss Light Source}: The \acs{sls} at the \acs{psi} is a third-generation synchrotron light source.
		\acro{snr}[SNR]{Signal-to-Noise Ratio}
		\acro{sr}[SR]{Synchrotron Radiation}
		\acro{srct}[SRCT]{Synchrotron Radiation Computed Tomography}
		\acro{srxtm}[SRXTM]{Synchrotron Radiation based X-ray Tomographic Microscopy}
		\acro{stl}[STL]{Surface Tessellation or Standard Triangulation Language or Standard Tessellation Language}: A file format used as a Quasi-Standard of many \acs{cad}-systems.
		\acro{vrml}[VRML]{Virtual Reality Modeling Language}
		\acro{wf-srxtm}[WF-SRXTM]{Wide Field Synchrotron Radiation Based X-ray Tomographic Microscopy}: The method to increase the field of view of tomographic imaging stations presented in \autoref{ch:haberthuer2010}.
		\acro{xml}[XML]{Extensible Markup Language}: A set of rules for electronical encoding of documents.
		\acro{yag}[YAG]{Yttrium Aluminium Garnet}
	\end{acronym}
\endgroup

\cleardoublepage