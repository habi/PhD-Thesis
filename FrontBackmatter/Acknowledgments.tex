% !TEX root = ../Thesis.tex
\acresetall
\pdfbookmark[1]{Acknowledgments}{acknowledgments}

\begingroup
\let\clearpage\relax
\let\cleardoublepage\relax
\let\cleardoublepage\relax
\chapter*{Acknowledgments}
I thank Prof.\ Dr.\ Johannes C.\ Schittny for giving me the opportunity to carry out this work at the \href{http://www.ana.unibe.ch/}{Institute of Anatomy} at the \href{http://unibe.ch/}{University of Bern}. Sharing his vast knowledge, Johannes introduced me into the delicate details of lung morphology, physiology and development and also provided multiple enjoyable moments away from the lab routine.

Dr.\ Miguel Gonzàlez helped me to get on track with various image processing intricacies and was---after changing job from the \href{http://www.istb.unibe.ch/}{\ac{istb}} in Bern to \href{http://www.alma3d.com/}{Alma IT Systems} in Spain---very well substituted by Dr.\ Mauricio Reyes from the \ac{istb}, which watched over the second half of my time working for this thesis. Both Miguel and Mauricio deserve thanks for their guidance as co-referees of this work. Prof.\ Dr.\ Martin Frenz acted as my mentor for the \href{http://www.gcb.unibe.ch/}{Graduate School for Cellular and Biomedical Sciences} of the University of Bern. Already during my Master thesis I had the pleasure to work under the guidance of  Martin and I am thankful for his watching eye as a mentor.

Both Sébastien Barré and Lilian Salm shared the office with me during the last months at the Institute of Anatomy and contributed to the pleasant atmosphere in our group. Sébastien was a helpful discussion partner for all MATLAB problems and during countless hours trying to stay awake and alert during numerous beamtime shifts at \href{http://sls.web.psi.ch/view.php/beamlines/tomcat/}{\acs{tomcat}}. Lilian spent many kyphotic\graffito{She's a med student, she knows what kyphotic means\ldots} hours counting alveolar bridges with the STEPanizer and is responsible for the nice counting results shown in the discussion. Mohammed Ouanella was an expert help with the preparation of the samples and also supplied excellent food from his home country, Algeria. Thanks!

The whole staff at the Institute of Anatomy provided for a great time; special thanks go to Dr.\ Stefan Tschanz for unconditional help with all things IT and for putting up with my special wishes concerning the use of the computing infrastructure here at the institute. Chrigu Lehmanns steady hand was welcome for the lathe machining \SI{0.6}{\milli\meter} Epon samples before we switched to paraffin embedded ``big'' lung samples. He also always provided generous help with special mechanical tasks, be it work related or not, \eg like a riveting a broken buckle back to my ski boots. Our secretary team and especially Therese Dudan always provided the first friendly smiles in the morning when I came in to pick up our mail.

Special thanks go to the whole team at the \acs{tomcat} beamline at the \href{http://sls.web.psi.ch/}{\acl{sls}} of the \href{http://psi.ch/}{\acl{psi}} in Villigen; without them, none of the pretty images in this thesis would exist. Prof.\ Dr.\ Marco F.\ M.\ Stampanoni, head of the \acs{tomcat} beamline at the \acl{sls} and assistant Professor for X-ray Microscopy at the \href{http://www.biomed.ee.ethz.ch/}{Institute for Biomedical Engineering} of the \href{http://ethz.ch/}{ETH in Zürich} provided a pleasurable work environment and was an invaluable resource on all things tomography and x-rays, provided well founded critique for my work and helped me to brush up my Italian skills both written and spoken. Dr.\ Christoph Hintermüller was not only a great guidance during my Master thesis for the postgraduate course in Medical Physics but provided expert support at the beamline and countless interesting discussions, both scientific and completely non-scientific. Xris' insight into scripting, UNIX commands and image processing was a great help. Without the work of Dr.\ Federica Marone on the tomographic reconstruction algorithms and implementation of those algorithms at \acs{tomcat} the wide-field-scan tomographic reconstructions in this work would not exist.

The \href{http://www.sport.unibe.ch/}{University sports Berne}, the \href{http://velokurierbern.ch/}{Velokurier Bern} and Fabio Breil with the eccentric bike provided me with some athletic balance to the time spent at the desk in front of my monitors.

My friends Pesche, Bruni, Wöufu, Sigi and \href{http://www.nicolafrombern.com/}{Nicola} as well as Mara and Barbara provided (sometimes much needed) diversion from the academic turf and shared loads of lunch times. Dan and Tobi as my former flat mates also helped to wrap my brain around different stuff than academics. Thanks!

Without the constant support from my family I would not be where I am today. Mami, Papi and Nina, I'll always be grateful to all of you.

Nina\graffito{Note that both my sister and girlfriend share the same first name, but are different persons. Having said that, I thank all the important Ninas in my life for the help with proofreading this work.}, you are the best sport teammate, friend, flatmate, reviewer, gardener, supporter and partner one could wish for. Your love makes me happy. I {\color{red}$\heartsuit$} you!
\endgroup