% !TEX root = ../Thesis.tex
\pagestyle{empty}
\hfill
\vfill
\section*{Colophon}
\refstepcounter{dummy}
\pdfbookmark[0]{Colophon}{Colophon}
\addcontentsline{toc}{chapter}{Colophon}
This thesis was typeset with \href{http://www.latex-project.org/}{\LaTeXe} using Hermann Zapf's \emph{Palatino} and \emph{Euler} type faces (Type~1 PostScript fonts \emph{URW Palladio L} and \emph{FPL} were used). The listings are typeset in \emph{Bera Mono}, originally developed by Bitstream, Inc. as ``Bitstream Vera''. (Type~1 PostScript fonts were made available by Malte Rosenau and Ulrich Dirr.)

The typographic style was inspired by \cauthor{bringhurst:2002}'s genius as presented in \emph{The Elements of Typographic Style} \citep{bringhurst:2002}. The \LaTeX\ package \href{http://www.ctan.org/tex-archive/macros/latex/contrib/classicthesis/}{\texttt{classicthesis}} has been made by \href{http://www.miede.de}{André Miede}. The colors of the package have been adapted for the \href{http://www.kommunikation.unibe.ch/intern/content/beratung/corporate_design/logo_schriften__farben/}{corporate design} of the \href{http://www.unibe.ch/}{University of Bern}.

\paragraph{note:} The custom size of the text block was calculated using the directions given by Mr.\ \citeauthor{bringhurst:2002} (pages 26--29 and 175/176). \SI{10}{pt} Palatino needs \SI{133.21}{pt} for the string ``abcdefghijklmnopqrstuvwxyz''. This yields a good line length between 24--\SI{26}{pc} (288--\SI{312}{pt}). Using a ``\emph{double square text block}'' with a 1:2 ratio this results in a text block of 312:\SI{624}{pt} (which includes the headline in this design). A good alternative would be the ``\emph{golden section text block}'' with a ratio of 1:1.62, here 312:\SI{505.44}{pt}. For comparison, \texttt{DIV9} of the \texttt{typearea} package results in a line length of \SI{389}{pt} (\SI{32.4}{pc}), which is by far too long. However, this information will only be of interest for hardcore pseudo-typographers like me.

%To make your own calculations, use the following commands and look up the corresponding lengths in the book:
%\begin{verbatim}
%	\settowidth{\abcd}{abcdefghijklmnopqrstuvwxyz}
%	\the\abcd\ % prints the value of the length
%\end{verbatim}
%Please see the file \texttt{classicthesis.sty} for some precalculated values for Palatino and Minion.
%
%\settowidth{\abcd}{abcdefghijklmnopqrstuvwxyz}
%\the\abcd\ % prints the value of the length

\bigskip

\noindent Typeset from \myVersion

\noindent \finalVersionString